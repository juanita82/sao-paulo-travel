\documentclass[]{article}
\usepackage{lmodern}
\usepackage{amssymb,amsmath}
\usepackage{ifxetex,ifluatex}
\usepackage{fixltx2e} % provides \textsubscript
\ifnum 0\ifxetex 1\fi\ifluatex 1\fi=0 % if pdftex
  \usepackage[T1]{fontenc}
  \usepackage[utf8]{inputenc}
\else % if luatex or xelatex
  \ifxetex
    \usepackage{mathspec}
    \usepackage{xltxtra,xunicode}
  \else
    \usepackage{fontspec}
  \fi
  \defaultfontfeatures{Mapping=tex-text,Scale=MatchLowercase}
  \newcommand{\euro}{€}
\fi
% use upquote if available, for straight quotes in verbatim environments
\IfFileExists{upquote.sty}{\usepackage{upquote}}{}
% use microtype if available
\IfFileExists{microtype.sty}{%
\usepackage{microtype}
\UseMicrotypeSet[protrusion]{basicmath} % disable protrusion for tt fonts
}{}
\usepackage[margin=1in]{geometry}
\ifxetex
  \usepackage[setpagesize=false, % page size defined by xetex
              unicode=false, % unicode breaks when used with xetex
              xetex]{hyperref}
\else
  \usepackage[unicode=true]{hyperref}
\fi
\hypersetup{breaklinks=true,
            bookmarks=true,
            pdfauthor={Ana I. Moreno-Monroy (Universitat Rovira i Virgili), Frederico R. Ramos (Fundação Getulio Vargas) and Robin Lovelace (University of Leeds)},
            pdftitle={The unequal distribution of access to education and transport: insights from for São Paulo},
            colorlinks=true,
            citecolor=blue,
            urlcolor=blue,
            linkcolor=magenta,
            pdfborder={0 0 0}}
\urlstyle{same}  % don't use monospace font for urls
\setlength{\parindent}{0pt}
\setlength{\parskip}{6pt plus 2pt minus 1pt}
\setlength{\emergencystretch}{3em}  % prevent overfull lines
\setcounter{secnumdepth}{0}

%%% Use protect on footnotes to avoid problems with footnotes in titles
\let\rmarkdownfootnote\footnote%
\def\footnote{\protect\rmarkdownfootnote}

%%% Change title format to be more compact
\usepackage{titling}

% Create subtitle command for use in maketitle
\newcommand{\subtitle}[1]{
  \posttitle{
    \begin{center}\large#1\end{center}
    }
}

\setlength{\droptitle}{-2em}
  \title{The unequal distribution of access to education and transport: insights
from for São Paulo}
  \pretitle{\vspace{\droptitle}\centering\huge}
  \posttitle{\par}
  \author{Ana I. Moreno-Monroy (Universitat Rovira i Virgili), Frederico R. Ramos
(Fundação Getulio Vargas) and Robin Lovelace (University of Leeds)}
  \preauthor{\centering\large\emph}
  \postauthor{\par}
  \date{}
  \predate{}\postdate{}



\begin{document}

\maketitle


\section{Abstract}\label{abstract}

In many large Latin American cities such as São Paulo, growing social
and economic inequalities are embedded through unfair education and
transport systems. Good schools are mostly concentrated in wealthy
areas, while transport links to school are deficient in deprived areas,
exacerbating the issue. Inequalities in educational and transport
infrastructure are mutually reinforcing: the right to mobility is
intrinsically linked to the right to education. This is manifested by
the overlap between recent protests against unwanted changes to public
education and the social movements contesting increases in public
transport fares. Another manifestation is to be found in the concept of
school accessibility. This paper sheds light on the transport-education
inequality nexus with reference to a new school accessibility measure
applied São Paulo. By capturing both the unequal distribution of schools
and transport services across space, the index allows embedded
inequalities to be better understood and, with political will,
contested. Our index combines information on the spatial distribution of
children and adolescents by income categories, the location of existing
schools, the travel patterns of students, and the travel infrastructure
serving the school catchment area into a single measure. The index is
used to measure school accessibility locally across São Paulo, using
data sources from Population and School Censuses, commonly available in
Latin American cities. The results illustrate how existing inequalities
are amplified by variable accessibility to schools across income groups
and geographical space. We conclude that extending the concept of local
accessibility indicators to education can help to both contest and
constructively tackle embedded social inequalities.

\section{Introduction}\label{introduction}

\textbf{Contents:}

\begin{itemize}
\itemsep1pt\parskip0pt\parsep0pt
\item
  Motivations
\item
  General measures of accessibility
\item
  School accessibility
\end{itemize}

\ldots{}

Travel to school is an everyday reality for millions of young people
around the world. The mode, duration, safety, comfort and pollution
levels of this trip has huge impacts on the future generation, yet has
received relatively little academic attention.

Travel options are vital for ensuring a more equitable supply of
educational opportunity to diverse groups. Conversely, poor
accessibility to deprived area can reinforce social inequalities, with
long-term implications. Based on this emerging context, this paper
develops a school accessibility index for local areas.

The first well-known attempt to define accessibility quantitatively was
by Ingram (1971), which presented a range of measures related to
distance (Euclidean and network), barriers and different functions
representing distance decay.

This early work made the distinction between accessibility indeces that
apply to zones or single `desire lines': ``relative accessibility is
defined as a measure of the effort of overcoming spatial separation
between two points, while the integral accessibility is defined as a
measure of the effort of overcoming spatial separation between a point
and all other points within an area'' (Allen, Liu, and Singer 1993).

\section{Method: measuring school
accessibility}\label{method-measuring-school-accessibility}

We calculate three different accessibility measures for different modes:
cumulative, gravitiy-based, and competitive.

The cumulative-opportunity measure for mode \(M\) (Boisjoly and
El-Geneidy, 2016) is defined as:

\[ CO_{i}^M= \sum_{j=1}^n(S_{j}f(C_{ij}) \]

\[ f(C_{ij}) = \left\{ 
                \begin{array}{ll}
                  1 if C_{ij}<=t\\
                  0 if C_{ij}<t
                \end{array}
              \right.
  \]

This measure counts the number of schools available from one area within
a certain travel time by mode \(M\) threshold. \(C_{ij}\) is the travel
cost (measured in time) between the centroid of zone i and the centroid
of zone j.

Gravity-based accessibility measure

In spite of its simplicity, Boisjoly and El-Geneidy (2016) have found
that the cumulative-opportunity measure (using a 45 minute threshold) is
highly correlated with a gravity-based accessibility measure
(({\textbf{???}}), ({\textbf{???}})), defined as:

\[ A_{i}^M= \sum_{j=1}^n(S_{j}e^{\beta*C_{ij}}) \]

Where \(A_{i}^M\) is the gravity accessibility at the centroid of area
\(i\) to all schools at area \(j\) using mode \(M\), and \(\beta\) is a
negative exponential cost function. The cost function, based on a
negative exponential decay curve, includes reported number of school
trips in the 2007 Origin-Destination Survey. (Here we can also use
information from the school census on number of students per school to
know how many students the area attracts). The proposed accessibility
measure captures the fact that more proximate schools (S) are more
attractive than those located further away.

Competitive accessibility measure

The measures proposed above take into account the sptail distribution of
schooling opportunities, but not the local demand for schooling. This is
particularly relevant for our equity analysis, since it could be the
case that higher areas are disproportionally served with respect to the
number of potentail students living within a certain travel distance,
whereas the opposite holds for lower-income areas. In order to assess
the mismatch between the demand and supply for schooling, we use the sum
of students in schools in each area (the demand), and the sum of
individuals within the school grade age-group living in each area (the
supply) in the following competitive accessibility measure, first
proposed by ({\textbf{???}}) and adapted by El-Geneidy et al (2015):

\[ CA_{i}^M= \sum_{j=1}^n\frac{P_{j}-f(C_{ij})}{\sum_{k=1}^n(Y_{k}-f(C_{kj}))}\]

Where the numerator discounts the number of pupils in area \(j\)
\(P_{j}\) by how far area \(i\) is from area \(j\) using the same
function as before, and the denominator discounts the number of students
(young adults) living in zone \(k\) \(Y_{k}\) by how far they are from
area \(j\). In this way, the discounted number of study places at each
area is divided by the potential students available to fill those
places, and then summed in order to obtain a single accessibility
measure for each area \(i\).

In order to assess equality in access to schooling, we stratify our
population of students in two student groups (we could do it for more
income groups): the students living with a head of household who earns 1
minimum wage or less, and those living with a head of household who
earns more than 1 minimum wage. The accessibility measures are based on
student's travel times by different modes using data from the 2007
Origin-Destination survey. Alternatively, for the case of public
transport, wet use actual transit times in 2015 for comparison (bilhete
unico data?).

\section{Results}\label{results}

\begin{itemize}
\itemsep1pt\parskip0pt\parsep0pt
\item
  Descriptive statistics about modal choice and travel time by AEPs and
  level of income
\item
  Spatial Distribution of Schools and Transit-Dependent Students at the
  AEP Level
\end{itemize}

\section{Discussion}\label{discussion}

\section{Conclusion}\label{conclusion}

\section{References}\label{references}

Allen, W Bruce, Dong Liu, and Scott Singer. 1993. ``Accesibility
Measures of US Metropolitan Areas.'' \emph{Transportation Research Part
B: Methodological} 27 (6). Elsevier: 439--49.

Ingram, David R. 1971. ``The Concept of Accessibility: A Search for an
Operational Form.'' \emph{Regional Studies} 5 (2). Taylor \& Francis:
101--7.

\end{document}
